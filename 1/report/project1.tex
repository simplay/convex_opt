\documentclass{paper}

%\usepackage{times}
\usepackage{epsfig}
\usepackage{graphicx}
\usepackage{amsmath}
\usepackage{amssymb}
\usepackage{color}
\usepackage{mathtools}


% load package with ``framed'' and ``numbered'' option.
%\usepackage[framed,numbered,autolinebreaks,useliterate]{mcode}

% something NOT relevant to the usage of the package.
\setlength{\parindent}{0pt}
\setlength{\parskip}{18pt}






\usepackage[latin1]{inputenc} 
\usepackage[T1]{fontenc} 

\usepackage{listings} 
\lstset{% 
   language=Matlab, 
   basicstyle=\small\ttfamily, 
} 

\newcommand{\norm}[1]{\left\lVert#1\right\rVert}


\title{Assignment 1}



\author{Single Michael\\08-917-445}
% //////////////////////////////////////////////////


\begin{document}



\maketitle


% Add figures:
%\begin{figure}[t]
%%\begin{center}
%\quad\quad   \includegraphics[width=1\linewidth]{ass2}
%%\end{center}
%
%\label{fig:performance}
%\end{figure}

\section{Demosaicing}
\subsection{Problem}
Most digital cameras acquire images from an image sensor overlaid with a color filter array (CFA). A color filter array depicts a mosaic of RGB color filters applied in front of the image sensor. An example is illustrated in figure $FIGURE$. This implies that each pixel in a raw camera image stores only the measured color channel value that corresponds to its pxiel-position in the CFA.




\subsection{Motivation}
$\min_{u \in X} \norm{\nabla u}_1 + \frac{\lambda}{2} \norm{Du - g}^2_2$

\subsection{Derivations}


\begin{equation}
	 \norm{\nabla{u_{c}}}_2 
	= \sum_{i=1}^N \sum_{j=1}^M \sqrt{\left( u_{c}\left(i+1, j\right) - u_{c}\left(i,j\right) \right)^2}
\end{equation}

\begin{equation}
	 \tau_{c}\left(i,j\right)
	= \sqrt{\left( u_{c}\left(i+1, j\right) - u_{c}\left(i,j\right) \right)^2}
\end{equation}

\begin{equation}
	\frac{\partial}{\partial u_{c}\left(i,j\right)} \norm{\nabla{u_c}}_2 = \frac{\partial{\tau_{c}\left(i,j\right)}}{\partial u_{c}\left(i,j\right)} + \frac{\partial{\tau_{c}\left(i-1,j\right)}}{\partial u_{c}\left(i,j\right)} + \frac{\partial{\tau_{c}\left(i,j-1\right)}}{\partial u_{c}\left(i,j\right)}
\end{equation}

\begin{align}
	\frac{\partial{\tau_{c}\left(i,j\right)}}{\partial u_{c}\left(i,j\right)}
	&= \frac{\partial}{\partial{u_{c}\left(i,j\right)}} \sqrt{ \left(u_{c}(i+1,j) - u_{c}(i,j)\right)^2 + \left(u_{c}(i,j+1) - u_{c}(i,j)\right)^2} \\
	&= \frac{1}{2} \frac{2 (u_{c}(i+1,j)-u_{c}(i,j))(-1) + 2 (u_{c}(i,j+1)-u_{c}(i,j))(-1)}{\sqrt{ \left(u_{c}(i+1,j) - u_{c}(i,j)\right)^2 + \left(u_{c}(i,j+1) - u_{c}(i,j)\right)^2}} \\
	&= \frac{1}{2} \frac{2 (-(u_{c}(i+1,j)-u_{c}(i,j)) - (u_{c}(i,j+1)-u_{c}(i,j))}{\tau_{c}\left(i,j\right)} \\
	&= \frac{2 u_{c} \left(i,j\right) - u_{c} \left(i+1,j\right)-u_{c} \left(i,j+1\right)}{\tau_{c}\left(i,j\right)}
\end{align}

\begin{align}
	\frac{\partial{\tau_{c}\left(i-1,j\right)}}{\partial u_{c}\left(i,j-1\right)}
	&= \frac{\partial}{\partial{u_{c}\left(i,j\right)}} \sqrt{ \left(u_{c}(i,j) - u_{c}(i-1,j)\right)^2 + \left(u_{c}(i-1,j+1) - u_{c}(i-1,j)\right)^2} \\
	&= \frac{1}{2} \frac{2(\left(u_{c}(i,j) - u_{c}(i-1,j)\right)(1+0) + \left(u_{c}(i-1,j+1) - u_{c}(i-1,j)\right)(0+0))}{\sqrt{ \left(u_{c}(i,j) - u_{c}(i-1,j)\right)^2 + \left(u_{c}(i-1,j+1) - u_{c}(i-1,j)\right)^2}} \\
	&= \frac{1}{2} \frac{2(\left(u_{c}(i,j) - u_{c}(i-1,j)\right)}{\tau_{c}\left(i,j\right)} \\
	&= \frac{\left(u_{c}(i,j)\right)-u_{c} \left(i-1,j\right)}{\tau_{c}\left(i,j\right)}
\end{align}

\begin{align}
	\frac{\partial{\tau_{c}\left(i,j-1\right)}}{\partial u_{c}\left(i,j\right)}
	&= \frac{\left(u_{c}(i,j)\right)-u_{c} \left(i,j-1\right)}{\tau_{c}\left(i,j\right)}
\end{align}



\begin{enumerate}
\item \textbf{Problem.}

\item \textbf{Motivations.} Describe the reasons and motivations behind this problem.
\item \textbf{Derivation of gradient.} In this section you should:

\begin{itemize}
\item Write the finite difference approximation of the objective function $E$.
\item Compute the gradient of the objective function $\nabla_uE$.  
\end{itemize}


\item \textbf{Implement gradient descent for demosaicing.} In this section you should:

\begin{itemize}
\item Show some images, as the the gradient method progresses iteration by iteration. Display the initial and the final image and 3 more images in between.
\end{itemize}

\item \textbf{Show images obtained by very high, very low and optimal $\lambda$.} In this section you should:

\begin{itemize}
\item Display 3 images with different $\lambda$ (very low, very high and optimal).
\item Describe the effect of $\lambda$ on the solution.
\end{itemize}

\item \textbf{ Find optimal $\lambda$.} In this section you should:

\begin{itemize}
\item Display the $SSD$ vs. $\lambda$ graph.
\item Describe the effect of $\lambda$ with respect to the $SSD$ between the ground truth and the solution image.
\end{itemize}


\end{enumerate}


 \end{document}
 
 