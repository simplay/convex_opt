\documentclass{paper}

%\usepackage{times}
\usepackage{epsfig}
\usepackage{graphicx}
\usepackage{amsmath}
\usepackage{amssymb}
\usepackage{color}
\usepackage{mathtools}


% load package with ``framed'' and ``numbered'' option.
%\usepackage[framed,numbered,autolinebreaks,useliterate]{mcode}

% something NOT relevant to the usage of the package.
\setlength{\parindent}{0pt}
\setlength{\parskip}{18pt}






\usepackage[latin1]{inputenc} 
\usepackage[T1]{fontenc} 

\usepackage{listings} 
\lstset{% 
   language=Matlab, 
   basicstyle=\small\ttfamily, 
} 

\newcommand{\norm}[1]{\left\lVert#1\right\rVert}
\DeclareMathOperator*{\argmin}{arg\,min}

\title{Assignment 1}



\author{Single Michael\\08-917-445}
% //////////////////////////////////////////////////


\begin{document}



\maketitle


% Add figures:
%\begin{figure}[t]
%%\begin{center}
%\quad\quad   \includegraphics[width=1\linewidth]{ass2}
%%\end{center}
%
%\label{fig:performance}
%\end{figure}

\section{Demosaicing}
\subsection{Problem}
Most digital cameras acquire images from an image sensor overlaid with a color filter array (CFA). A color filter array depicts a mosaic of RGB color filters applied in front of the image sensor. An example is illustrated in figure $FIGURE$. This implies that each pixel in a raw camera image stores only the measured color channel value that corresponds to its pxiel-position in the CFA.


\subsection{Motivation}

\begin{align}
	\widetilde{u_R} = \argmin_{u_R} \norm{\nabla u_R}_2 + \frac{\lambda}{2} \norm{u_R - g}^2_{\Omega_{R}} \\
	\widetilde{u_G} = \argmin_{u_G} \norm{\nabla u_G}_2 + \frac{\lambda}{2} \norm{u_G - g}^2_{\Omega_{G}}\\
	\widetilde{u_B} = \argmin_{u_B} \norm{\nabla u_R}_2 + \frac{\lambda}{2} \norm{u_B - g}^2_{\Omega_{B}}		
\end{align}

\begin{equation}
	\norm{u_C - g}^2_{\Omega_{C}} = \sum_x \sum_y \Omega_{C}(x,y)\norm{u_{c}(x,y) - g(x,y)}^2
\end{equation}

\begin{align}
	E(u_{c}) =  \norm{\nabla u_c}_2 + \frac{\lambda}{2} \norm{u_c - g}^2_{\Omega_{R}}
\end{align}

\subsection{Gradient Descend}

\begin{align}
	u_{c}(i,j) = u_{c}(i,j) - \alpha \frac{\partial{E}}{\partial{u_{c} (i,j)}}(i,j)
\end{align}

\subsection{Derivations}

In this subsection we derive an explicit approximation for the the partial derivative of our energy term $E$ that can be numerically computed. Such an approximation is useful since we will need its expression later for our update rule to solve our demosaicing optimization problem. For the discretization we will relying on a forward difference scheme. In other by the end of this section we will have derived an discretization of the following expression $\frac{\partial{E}}{\partial{u_{c} (i,j)}}(i,j)$. \\

% TODO: glue those two paragraphs somehow

using a forward difference approximation scheme for the derivatives $\partial_{i} u_c$ and $\partial_{j} u_c$ respectively, we can derive the following expression for the regularization term:

\begin{align}
	 \norm{\nabla{u_{c}}}_2 
	&= \sum_{i=1}^N \sum_{j=1}^M \sqrt{[\partial_{i} u_c(i,j)]^2 + [\partial_{j} u_c(i,j)]^2} \nonumber \\ 
	&= \sum_{i=1}^N \sum_{j=1}^M \sqrt{\left[ u_{c}\left(i+1, j\right) - u_{c}\left(i,j\right) \right]^2 + \left[ u_{c}\left(i, j+1 \right) - u_{c}\left(i,j\right) \right]^2}
\label{eq:regularization_expr}
\end{align}

In oder to simplify later derivation steps, let us introduce the helper function $\tau$ defined as the following 

\begin{equation}
	 \tau_{c}\left(i,j\right)
	= \sqrt{\left[ u_{c}\left(i+1, j\right) - u_{c}\left(i,j\right) \right]^2 + \left[ u_{c}\left(i, j+1 \right) - u_{c}\left(i,j\right) \right]^2}
\end{equation}

We notice that $\tau$ is only the expression under the summation in the right hand side in equation $\ref{eq:regularization_expr}$. Next, let us take the partial derivative along $u_{c}(i,j)$ of $\norm{\nabla{u_{c}}}_2$. By fixing a index pair ($i$ $j$) and reordering the terms under the summation, we see that 

\begin{equation}
	\frac{\partial}{\partial u_{c}\left(i,j\right)} \norm{\nabla{u_c}}_2 = \underbrace{\frac{\partial{\tau_{c}\left(i,j\right)}}{\partial u_{c}\left(i,j\right)}}_{\bf{(a)}} + \underbrace{\frac{\partial{\tau_{c}\left(i-1,j\right)}}{\partial u_{c}\left(i,j\right)}}_{\bf{(b)}} + \underbrace{\frac{\partial{\tau_{c}\left(i,j-1\right)}}{\partial u_{c}\left(i,j\right)}}_{\bf{(c)}}
\end{equation}

Next I will derive explicit expressions for the terms $\bf{(a)}$, $\bf{(b)}$ and $\bf{(c)}$. Loosely speaking our goal is to get rid of the partial derivative operator. The mathematical key concept I will rely on is the chain rule for partial derivatives. More precisely we will make use of the fact that $\partial_{x}\sqrt{f(x)}$ is $\frac{\partial_{x} f(x)}{2 \sqrt{f(x)}}$.

\begin{align}
	(a) \Longrightarrow
	& \frac{\partial{\tau_{c}\left(i,j\right)}}{\partial u_{c}\left(i,j\right)} \nonumber \\
	&= \frac{\partial}{\partial{u_{c}\left(i,j\right)}} \sqrt{ \left[u_{c}(i+1,j) - u_{c}(i,j)\right]^2 + \left[u_{c}(i,j+1) - u_{c}(i,j)\right]^2} \nonumber \\
	&= \frac{1}{2} \frac{2 (u_{c}(i+1,j)-u_{c}(i,j))(0-1) + 2 (u_{c}(i,j+1)-u_{c}(i,j))(0-1)}{\sqrt{ \left[u_{c}(i+1,j) - u_{c}(i,j)\right]^2 + \left[u_{c}(i,j+1) - u_{c}(i,j)\right]^2}} \nonumber \\
	&= \frac{1}{2} \frac{2 [-(u_{c}(i+1,j)-u_{c}(i,j)) - (u_{c}(i,j+1)-u_{c}(i,j)]}{\tau_{c}\left(i,j\right)} \nonumber \\
	&= \frac{2 u_{c} \left(i,j\right) - u_{c} \left(i+1,j\right)-u_{c} \left(i,j+1\right)}{\tau_{c}\left(i,j\right)}
\end{align}

\begin{align}
(b) \Longrightarrow
	& \frac{\partial{\tau_{c}\left(i-1,j\right)}}{\partial u_{c}\left(i,j\right)} \nonumber \\
	&= \frac{\partial}{\partial{u_{c}\left(i,j\right)}} \sqrt{ \left[u_{c}(i,j) - u_{c}(i-1,j)\right]^2 + \left[u_{c}(i-1,j+1) - u_{c}(i-1,j)\right]^2} \nonumber \\
	&= \frac{1}{2} \frac{2[\left(u_{c}(i,j) - u_{c}(i-1,j)\right)(1-0) + \left(u_{c}(i-1,j+1) - u_{c}(i-1,j)\right)(0-0)]}{\sqrt{ \left[u_{c}(i,j) - u_{c}(i-1,j)\right]^2 + \left[u_{c}(i-1,j+1) - u_{c}(i-1,j)\right]^2}} \nonumber \\
	&= \frac{1}{2} \frac{2\left[(u_{c}(i,j) - u_{c}(i-1,j)\right]}{\tau_{c}\left(i-1,j\right)} \nonumber \\
	&= \frac{u_{c}\left(i,j\right) - u_{c}\left(i-1,j\right)}{\tau_{c}\left(i-1,j\right)}
\end{align}

\begin{align}
(c) \Longrightarrow
	& \frac{\partial{\tau_{c}\left(i,j-1\right)}}{\partial u_{c}\left(i,j\right)} \nonumber \\
	&= \frac{\partial}{\partial{u_{c}\left(i,j\right)}} \sqrt{ \left[u_{c}(i+1,j-1)-u_{c}(i,j-1)\right]^2 + \left[u_{c}(i,j) - u_{c}(i,j-1)\right]^2} \nonumber \\
	&=  \frac{1}{2} \frac{2 [\left(u_{c}(i+1,j-1)-u_{c}(i,j-1)\right)(0 - 0) + \left(u_{c}(i,j) - u_{c}(i,j-1)\right)(1-0)]}{\sqrt{ \left[u_{c}(i+1,j-1)-u_{c}(i,j-1)\right]^2 + \left[u_{c}(i,j) - u_{c}(i,j-1)\right]^2}} \nonumber \\
	&= \frac{1}{2} \frac{2 \left[u_{c}(i,j)- u_{c} (i,j-1)\right] }{\tau_{c}\left(i,j-1\right)} \nonumber \\			
	&= \frac{u_{c}(i,j)-u_{c} \left(i,j-1\right)}{\tau_{c}\left(i,j-1\right)}
\end{align}



\begin{enumerate}
\item \textbf{Problem.}

\item \textbf{Motivations.} Describe the reasons and motivations behind this problem.
\item \textbf{Derivation of gradient.} In this section you should:

\begin{itemize}
\item Write the finite difference approximation of the objective function $E$.
\item Compute the gradient of the objective function $\nabla_uE$.  
\end{itemize}


\item \textbf{Implement gradient descent for demosaicing.} In this section you should:

\begin{itemize}
\item Show some images, as the the gradient method progresses iteration by iteration. Display the initial and the final image and 3 more images in between.
\end{itemize}

\item \textbf{Show images obtained by very high, very low and optimal $\lambda$.} In this section you should:

\begin{itemize}
\item Display 3 images with different $\lambda$ (very low, very high and optimal).
\item Describe the effect of $\lambda$ on the solution.
\end{itemize}

\item \textbf{ Find optimal $\lambda$.} In this section you should:

\begin{itemize}
\item Display the $SSD$ vs. $\lambda$ graph.
\item Describe the effect of $\lambda$ with respect to the $SSD$ between the ground truth and the solution image.
\end{itemize}


\end{enumerate}


 \end{document}
 
 